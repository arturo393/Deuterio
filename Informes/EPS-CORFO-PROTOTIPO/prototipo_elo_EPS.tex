%%%%%%%%%%%%%%%%%%%%%%%%%%%%%%%%%%%%%%%%%%%%%%%%%%%%%%%%%%%%%%%%%%%%%%%%%%%
% % Plantilla para un artculo en LaTeX en espaol.  %
%%%%%%%%%%%%%%%%%%%%%%%%%%%%%%%%%%%%%%%%%%%%%%%%%%%%%%%%%%%%%%%%%%%%%%%%%%%

\documentclass[12pt,twoside,onecolumn]{article}

% Esto es para poder escribir acentos directamente:
%\usepackage[latin1]{inputenc} 
\usepackage[utf8]{inputenc}
\usepackage{makeidx} \usepackage{multirow} \usepackage{titlesec}
\usepackage{sectsty} \usepackage{fncychap}

%\usepackage{natbib}

\renewcommand{\thesection}{}
\renewcommand{\thesubsection}{\arabic{subsection}}
% Esto es para que el LaTeX sepa que el texto est en espaol: 
\usepackage[spanish]{babel}
\usepackage[right=3cm,left=3cm,top=2.5cm,bottom=2.5cm,headsep=1cm,footskip=2cm]{geometry}
\usepackage{graphicx}
% Paquetes de la AMS: 
\usepackage{amsmath, amsthm, amsfonts}
\renewcommand{\thesection}{}
\renewcommand{\thesubsection}{\arabic{subsection}}
\usepackage{hyperref} 
%\pagestyle{headings} 
\pagestyle{myheadings}
\markboth{Documentación Prototipo EPS}{}

\begin{document} 
\setlength{\unitlength}{1 cm} %Especificar unidad de trabajo
\thispagestyle{empty}
%\begin{picture}(25,8)
%\includegraphics[width=14cm,height=9cm]{deu2.jpg}
%\end{picture}

\vspace*{-1in}
\begin{figure}[htb]
\begin{center}
\includegraphics[width=12cm]{./figures/deutlogo.png}
\end{center}
\end{figure}

\begin{center}
\textbf{
{\LARGE Departamento de Electr\'onica}}\\[1.25cm]
{\Large Documento de Dise\~no}\\[1.3cm]
{\LARGE \textbf{Operaci\'on Big Bobbin}}\\[2.5cm]
{\large Arturo Veras Olivos}\\[1cm]
Valparaiso - \today
\\[2cm]
{\texttt{\textit{Water eventually would be employed as a fuel and that the hydrogen and oxygen which constitute it would furnish an inexhaustible source of heat and light. Jules Verne}}}
\end{center}
 
%\newpage 
%\tableofcontents
%\listoffigures % to produce list of figures 
%\listoftables % to producelist of tables 
%\begin{abstract}
%   abstract-text
%\end{abstract}
\newpage 
\section{Alcance}
Este documento da cuenta del trabajo realizado para el proyecto 15-COTE42665 en relación a la electrónica del prototipo.

\section{Introducción}
%% Explicación de lo que la electrónica hará para el protitpo EPS , dentro de lo que se dijo que iba a hacer.

\section{Resultado 1 - Diseño Prototipo EPS (Abril)} %% Métrica: Diseño y  planos de construcción.
%% Explicación con diagrama de lo que hace el circtuito y sus características más imporantantes (P máx, formas de onda, seguridad)
%% Identifación de partes dentro del diagrama y sus eficiencias.

\subsection{DC Link}
%% Mostrar forma de onda y potencia de salida.
%% Diagrama del circuito más anexo con PCB.

\subsection{Disparo}
%% Mostrar forma de onda y potencia de salida.
%% Diagrama del circuito más anexo con PCB.

\subsection{Sensores}
%% Indicar sensores (el modelo en caso sea ), características de medición (raango de medición). 


\section{Resultado 2 - Construcción EPS (Junio)} %% Métrica:Producto físico
%% Imagen con el rack diseñado, ojalá construido y explicando sus partes. 

\section{Resultado 3 - Puntos Óptimos de Operación EPS (Diciembre)} %% Métrcica frecuencia en Hz
%% Consideraciónes de funcionamiento, ej: temperatura, corriente , Potencia , frecuencia , tiempo de funcionamiento y comportamiento en estas condiciones. 

\newpage
\bibliographystyle{plain}

\bibliography{Bibliography}{}

\end{document}

